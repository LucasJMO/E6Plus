\textbf{\huge{Psion Uncarnate}}

\textbf{Hit Die}: d4

\textbf{Skills}: Autohypnosis (Wis), Bluff (Cha), Concentration (Con), Craft (any) (Int), Disguise (Cha), Knowledge (the planes) (Int), Knowledge (psionics) (Int), Psicraft (Int), Sense Motive (Wis)

\textbf{Skills/Level}: 2 + Int modifier

\begin{center}
\begin{adjustwidth}{-4cm}{}
\begin{small}
\begin{tabular}{| c | c | c | c | c | c | c | c | c |}
\hline
LVL &BAB &F &R &W &Special &Power Points/Day &Powers Known &Maximum Power Level Known \\
\hline
1 &0 &0 &0 &2 &Incorporeal touch, uncarnate armor &2 &1 &1st \\
2 &1 &0 &0 &3 &Shed body &6 &2 &1st \\
3 &1 &1 &1 &3 &Assume equipment, assume likeness &10 &3 &1st \\
4 &2 &1 &1 &4 &Uncarnate shell &15 &4 &2nd \\
5 &2 &1 &1 &4 &Telekinetic force, uncarnate bridge &19 &5 &2nd \\
6 &3 &2 &2 &5 &Uncarnate &26 &6 &2nd \\
\hline
\end{tabular}
\end{small}
\end{adjustwidth}
\end{center}

\textbf{Weapon and Armor Proficiency}: Psion uncarnates are proficient with light armor.

\textbf{Power Points/Day}: A psion uncarnate's ability to manifest powers is limited by the power points he has available. His base daily allotment of power points is given on the table above. In addition, he receives bonus power points per day if he has a high Wisdom score. His race may also provide bonus power points per day, as may certain feats and items. A 1st-level psion uncarnate gains no power points for his class level, but he gains bonus power points (if he is entitled to any), and can manifest the single power he knows with those power points.

\textbf{Powers Known}: A psion uncarnate begins play knowing one psion power of your choice. Each time he achieves a new level, he unlocks the knowledge of a new power.

Choose the powers known from the psion power list. A psion uncarnate can manifest any power that has a power point cost equal to or lower than his manifester level.

The total number of powers a psion uncarnate can manifest in a day is limited only by his daily power points.

A psion uncarnate simply knows his powers; they are ingrained in his mind. He does not need to prepare them (in the way that some spellcasters prepare their spells), though he must get a good night's sleep each day to regain all his spent power points.

The Difficulty Class for saving throws against psion uncarnate powers is 10 + the power's level + the psion uncarnate's Wisdom modifier.

\textbf{Incorporeal Touch(Su)}: Beginning at 1st level, a psion uncarnate can make melee touch attacks that deal 1d6 points of damage per two class levels (rounded down) if they hit. The character's Strength modifier is not applied to this attack, but it is effective against incorporeal creatures (and against corporeal creatures while the psion uncarnate is incorporeal) The character's hand and arm seem to become slightly translucent when he makes these attacks. 

\textbf{Uncarnate Armor}: A psion uncarnate wearing armor (or using inertial armor or a similar effect) gets his armor bonus to AC even when he becomes incorporeal (see Shed Body, below). However, unlike other incorporeal creatures, a psion uncarnate does not gain a deflection bonus to Armor Class from his Charisma modifier. This ability works even if the armor being worn becomes incorporeal (such as through the use of the assume equipment ability described below).

\textbf{Shed Body}: Starting at 2nd level, a psion uncarnate can become incorporeal (or "uncarnate") once per day per level as a standard action. The character can remain uncarnate for up to 1 minute. During this time, the character's body fades into an immaterial form that retains the character's basic likeness. While uncarnate, the character gains the incorporeal subtype (see below). His material armor remains in place and continues to provide its armor bonus to AC (see Uncarnate Armor, above). His material weapons also remain corporeal. Losing his physical form allows the character to more easily access his mental abilities, and he gains a +1 bonus on all save DCs for powers he manifests while uncarnate.

He can use equipment normally, deriving benefits from items that enhance his capabilities; however, all his equipment remains material even when the character is uncarnate (but see the assume equipment ability, described below).

Often, a psion uncarnate appears almost like a ghost wearing items of the material world. This doesn't make his equipment more susceptible to attack (the normal rules for attended objects apply), but it does make it impossible for the character to enter or pass through solid objects (or use his uncarnate shell ability) while wearing solid equipment. If he drops his material equipment, he can pass through solid objects at will as described below.

While using this ability the psion uncarnate gains the incorporeal subtype and all of its associated abilities except:
* Unlike other incorporeal creatures, a psion uncarnate does not gain a deflection bonus from his Charisma modifier.
* A psion uncarnate does not gain any other forms of movement, but floats just off the ground. While he remains within 1 foot of a flat surface of any solid or liquid, he can take normal actions, make normal attacks, and move at his normal speed (he can even "run" at up to four times his normal speed). At distances higher than 1 foot he falls, but may slow his speed to a mere 60 feet per round (if he chooses), and does not take fall damage.
* A psion uncarnate who is submerged in water may move his base speed undaunted in any direction he chooses.

\textbf{Assume Equipment}: Beginning at 3rd level, a psion uncarnate can designate a number of pieces of his worn equipment (including armor and weapons) equal to his psion uncarnate level to become incorporeal when he uses his shed body ability. This has no effect on the equipment's function, but now when the psion uncarnate is incorporeal, he can enter or pass through solid objects while wearing nothing other than the designated equipment. Once designated, the equipment automatically changes to incorporeal when the character sheds his body, and it returns to corporeality when the character does. The character can change his designations as he desires.

\textbf{Assume Likeness}: At 3rd level and higher, while incorporeal, a psion uncarnate can assume the likeness of any Small, Medium, or Large creature as a standard action that does not provoke attacks of opportunity. The character's abilities do not change, but he appears to be the creature that he assumes the likeness of, allowing him the ability to effectively disguise himself and bluff those who might wonder at his true nature. Each physical interaction with a creature requires a successful Bluff check (opposed by the creature's Sense Motive check) to convince the creature of the psion uncarnate's new appearance. The psion uncarnate must not do anything to give away his true (incorporeal) nature in order for the bluff to be successful; for instance, if he accepts an item from another creature only to have it fall through his immaterial hands, the Bluff check automatically fails. However, a Bluff check would be allowed if the psion uncarnate uses his telekinetic force ability (see below) to hold the received item.

\textbf{Uncarnate Shell(Su)}: Beginning at 4th level, while incorporeal, a psion uncarnate can enter the body of a willing being. The psion uncarnate must be in a space adjacent to the creature to be entered, and expend a standard action. In order for a creature to be a valid target for this ability it most not be undead, not be incorporeal, not be a construct, and have an Intelligence score.

The psion uncarnate and its shell creature can communicate telepathically. If the shell creature needs to make a Will saving throw he may use the psion uncarnate's save instead of his own. Additionally the psion uncarnate may make any mental (Intelligence, Wisdom, or Charisma based) skill checks for the shell creature. Use of uncarnate shell is not noticeable with only the naked eye, however use of the detect psionics power will reveal the psion uncarnate.

During combat the shell creature and the psion uncarnate roll initiative separately. The shell creature acts normally. On the psion uncarnate's turn he may use a single standard action to either use one of his powers, his telekinetic force ability, or uncarnate bridge (see below). If the psion uncarnate acts before the shell creature then the shell creature is no longer considered flat footed.

Once inside the shell creature the psion uncarnate cannot be the target of attacks or effects and cannot take any form of damage. However if the shell creature suffers any damage the psion uncarnate must make a concentration check of DC 5 + damage dealt or the effect ends and the psion uncarnate is ejected from the shell creature.

The effect ends if the concentration check described above is failed, the psion uncarnate's use of shed body expires, the shell creature dies, the shell creature chooses to eject the psion uncarnate (a free action), or the psion uncarnate chooses to end the effect (a free action). When this occurs, the psion uncarnate is shunted into an open space adjacent to the shell creature. If there are no adjacent open spaces then the psion uncarnate is placed in the nearest open space to the shell creature and suffers 1d6 damage for each 5 feet (after the first) traveled this way.

\textbf{Telekinetic Force}: Beginning at 5th level, while incorporeal, a psion uncarnate can use a telekinetic force effect as a standard action that does not provoke attacks of opportunity. The save DC is equal to 14 + the psion uncarnate's Wisdom modifier. The character's manifester level is the manifester level of the effect.

Even while corporeal, a psion uncarnate can use this ability, but only three times per day (uses while he is uncarnate do not count against this use limit).

\textbf{Uncarnate Bridge}: At 5th level, as a creature of almost pure mind, a psion uncarnate becomes more closely attuned to the minds of other creatures. He gains the ability to transport himself via the minds of living creatures. Once per day as a standard action while incorporeal, he can seamlessly enter any living creature with an Intelligence score and pass to another living creature with an Intelligence score that is within line of sight of the first creature.

The psion uncarnate must be in a space adjacent to the entry creature before transporting, and he appears in a space adjacent to the destination creature after transporting. Alternatively a shell creature (gained via the uncarnate shell ability) may be used as the entry creature. The entry and destination creatures need not be familiar to the character. A psion uncarnate cannot use himself as the entry or destination creature. Neither creature need be a willing participant.

When exiting the destination creature, the psion uncarnate chooses an adjacent square in which to appear. Entering and leaving a creature is painless, unless the psion uncarnate wishes otherwise (see below). In most cases, though, the destination creature finds being the endpoint of a mental bridge surprising and quite unsettling.

If he desires, a psion uncarnate can destructively exit the destination creature. If the creature fails a Will save (DC 13 + psion uncarnate's Wisdom modifier), the exiting psion uncarnate tunes his mental form to destructively interfere with the target's mind. He bursts forth explosively from the creature's body, dealing it 6d6 points of damage.

\textbf{Uncarnate(Ex)}: At 6th level, a psion uncarnate becomes a being of pure psionic consciousness. This ability functions like shed body, except the character is permanently incorporeal (and gains that subtype). If the character desires, he can become corporeal once per day for up to 1 minute, but he spends the rest of his time as an entity of mind untethered by the physical world.