\textbf{\huge{Ranger}}

\textbf{Hit Die}: d10

\textbf{Skills}: Balance (Dex), Climb (Str), Concentration (Con), Craft (Int), Handle Animal (Cha), Heal (Wis), Hide (Dex), Jump (Str), Knowledge (dungeoneering) (Int), Knowledge (geography) (Int), Knowledge (nature) (Int), Listen (Wis), Move Silently (Dex), Profession (Wis), Ride (Dex), Search (Int), Spot (Wis), Survival (Wis), Swim (Str), Tumble (Dex), Use Rope (Dex)

\textbf{Skills/Level}: 6 + Int modifier

\begin{center}
\begin{adjustwidth}{-4cm}{}
\begin{small}
\begin{tabular}{| c | c | c | c | c | c | c | c |}
\hline
LVL &BAB &F &R &W &Special &1 &2 \\
\hline
1 &1 &2 &2 &0 &1st Favored Enemy, Track, Wild Empathy, Point Blank Shot &- &- \\
2 &2 &3 &3 &0 &Combat Style TWF, Combat Style Ranged &- &- \\
3 &3 &3 &3 &1 &Endurance, 2nd Favored Enemy, Quick Draw &- &- \\
4 &4 &4 &4 &1 &Animal Companion &1 &- \\
5 &5 &4 &4 &1 &3rd Favored Enemy &2 &1 \\
6 &6/1 &5 &5 &2 &Improved Combat Style TWF, Improved Combat Style Ranged &3 &2 \\
\hline
\end{tabular}
\end{small}
\end{adjustwidth}
\end{center}

\textbf{Weapon and Armor Proficiency}: A ranger is proficient with all simple and martial weapons, and with light armor and shields (except tower shields).

\textbf{Spells}: A ranger casts divine spells, which are drawn from the ranger spell list. A ranger must choose and prepare her spells in advance. 

To prepare or cast a spell, a ranger must have a Wisdom score equal to at least 10 + spell level. The Difficulty Class for a saving throw against a ranger's spell is 10 + spell level + Wisdom modifier. 

\textbf{Favored Enemy(Ex)}: At 1st level, a ranger may select a type of creature from among those given on the table. The ranger gains a +2 bonus on Bluff, Listen, Sense Motive, Spot, and Survival checks when using these skills against creatures of this type. Likewise he gets a +2 bonus on weapon damage rolls, and a +1 to crit range against such creatures.

At 3rd and 5th level the ranger selects an additional favored enemy type from those given on the table. In addition, at each such interval, the bonuses against any one favored enemy (including the one just selected, if so desired) increase by 2 (save for the bonus to crit range, which increases by 1).

\begin{center}
\begin{adjustwidth}{-4cm}{}
\begin{small}
\begin{tabular}{| l | l |}
\hline
Type (Subtype) &Type (Subtype) \\
\hline
Aberration &Humanoid (reptilian) \\
animals &Magical beast \\
Construct &Monstrous Humanoid \\
Dragon &Ooze \\
Elemental &Outsider (air) \\
Fey &Outsider (chaotic) \\
Giant &Outsider (evil) \\
Humanoid (aquatic) &Outsider (fire) \\
Humanoid (dwarf) &Outsider (good) \\
Humanoid (elf) &Outsider (lawful) \\
Humanoid (goblinoid) &Outsider (native) \\
Humanoid (gnoll) &Outsider (water) \\
Humanoid (halfling) &Plant \\
Humanoid (human) &Undead \\
Humanoid (orc) &Vermin \\
\hline
\end{tabular}
\end{small}
\end{adjustwidth}
\end{center}

\textbf{Track}: A 1st level ranger gains Track as a bonus feat.

\textbf{Wild Empathy(Ex)}: A ranger can improve the attitude of an animal. This ability functions just like a Diplomacy check to improve the attitude of a person. The ranger rolls a d20 and adds his ranger level and his Charisma modifier to determine the wild empathy check result. The typical domestic animal has a starting attitude of indifferent, while wild animals are usually unfriendly.

To use wild empathy, the ranger and the animal must be able to study each other, which means that they must be within 30 ft of one another under normal visibility conditions. Generally influencing an animal in this way takes 1 minte, but, as with influencing people, it might take more or less time.

The ranger can also use this ability to influence a magical beast with an intelligence score of 1 or 2, but he takes a -4 penalty on the check.

\textbf{Point Blank Shot}: A 1st level ranger gains the feat Point Blank Shot, if he already has this feat then he may select another feat for which he meets the prerequisites.

\textbf{Combat Style(Ex)}: A 2nd level ranger selects either archery or two-weapon combat as his combat style. A ranger who selects archery gains the feat Rapid Shot, while a ranger who selects two-weapon combat gains the feat Two-Weapon Fighting. These feats may be gained even if the ranger does not meet the normal prerequisites for the chosen feat. The benefit of the ranger's chosen style apply only when he wears light armor or no armor. He loses all benefits of his combat style while wearing medium or heavy armor.

\textbf{Endurance}: A ranger gains endurance as a bonus feat at 3rd level. If they already have this feat they may select another feat for which they meet the prerequisites. 

\textbf{Quick Draw(Ex)}: A ranger may change from bow to melee weapons (or vice-versa) as a swift action. 

\textbf{Animal Companion(Ex)}: At 4th level, a ranger gains an animal companion selected from the following list: badger, camel, dire rat, dog, riding dog, eagle, hawk, horse (light or heavy), owl, pony, snake (Small or Medium viper), or wolf. If the campaign takes place wholly or partly in an aquatic environment, the following creatures may be added to the ranger’s list of options: manta ray, porpoise, Medium shark, and squid. This animal is a loyal companion that accompanies the ranger on his adventures as appropriate for its kind.

This ability functions like the druid ability of the same name, except that the ranger’s effective druid level is one-half his ranger level. A ranger may select from the alternative lists of animal companions just as a druid can, though again his effective druid level is half his ranger level. Like a druid, a ranger cannot select an alternative animal if the choice would reduce his effective druid level below 1st.

\textbf{Improved Combat Style(Ex)}: A 6th level ranger who selected the archery combat style gains Manyshot, while a 6th level ranger who selected two-weapon combat gains Improved Two-Weapon Fighting. These feats can be gained through Improved Combat Style even if the Ranger does not meet the normal prerequisites. The benefit of the ranger's chosen style apply only when he wears light armor or no armor. He loses all benefits of his combat style while wearing medium or heavy armor.