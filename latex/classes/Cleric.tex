\textbf{\huge{Cleric}}

\textbf{Hit Die}: d8

\textbf{Skills}: Concentration (Con), Craft (Int), Diplomacy (Cha), Heal (Wis), Knowledge (arcana) (Int), Knowledge (history) (Int), Knowledge (religion) (Int), Knowledge (the planes) (Int), Profession (Wis), and Spellcraft (Int)

\textbf{Skills/Level}: 2 + Int modifier

\begin{center}
\begin{adjustwidth}{-4cm}{}
\begin{small}
\begin{tabular}{| c | c | c | c | c | c | c | c | c | c |}
\hline
LVL &BAB &F &R &W &Special &0 &1 &2 &3 \\
\hline
1 &0 &2 &0 &2 &Turn or rebuke undead &3 &1+1 &- &- \\
2 &1 &3 &0 &3 & &4 &2+1 &- &- \\
3 &1 &3 &1 &3 & &4 &2+1 &1+1 &- \\
4 &2 &4 &1 &4 &Bonus Feat &5 &3+1 &2+1 &- \\
5 &2 &4 &1 &4 & &5 &3+1 &2+1 &1+1 \\
6 &3 &5 &2 &5 & &5 &3+1 &3+1 &2+1 \\
\hline
\end{tabular}
\end{small}
\end{adjustwidth}
\end{center}

\textbf{Weapon and Armor Proficiency}: Clerics are proficient with all simple weapons, with all types of armor (light, medium, and heavy), and with shields (except tower shields).

A cleric who chooses the War domain receives the Weapon Focus feat related to his deity’s weapon as a bonus feat. He also receives the appropriate Martial Weapon Proficiency feat as a bonus feat, if the weapon falls into that category.

\textbf{Aura}: A cleric of a chaotic, evil, good, or lawful deity has a particularly powerful aura corresponding to the deity’s alignment (see the detect evil spell for details). Clerics who don’t worship a specific deity but choose the Chaos, Evil, Good, or Law domain have a similarly powerful aura of the corresponding alignment.

\textbf{Spells}: A cleric casts divine spells, which are drawn from the cleric spell list. To prepare or cast a spell, a cleric must have a Wisdom score equal to at least 10 + the spell level. The Difficulty Class for a saving throw against a cleric’s spell is 10 + the spell level + the cleric’s Wisdom modifier.

Like other spellcasters, a cleric can cast only a certain number of spells of each spell level per day. His base daily spell allotment is given on the above table. In addition, he receives bonus spells per day if he has a high Wisdom score. A cleric also gets one domain spell of each spell level he can cast, starting at 1st level. When a cleric prepares a spell in a domain spell slot, it must come from one of his two domains (see Deities, Domains, and Domain Spells, below).

Clerics meditate or pray for their spells. Each cleric must choose a time at which he must spend 1 hour each day in quiet contemplation or supplication to regain his daily allotment of spells. Time spent resting has no effect on whether a cleric can prepare spells. A cleric may prepare and cast any spell on the cleric spell list, provided that he can cast spells of that level, but he must choose which spells to prepare during his daily meditation.

\textbf{Deity, Domains, and Domain Spells}: A cleric’s deity influences his alignment, what magic he can perform, his values, and how others see him. A cleric chooses two domains from among those belonging to his deity. A cleric can select an alignment domain (Chaos, Evil, Good, or Law) only if his alignment matches that domain.

If a cleric is not devoted to a particular deity, he still selects two domains to represent his spiritual inclinations and abilities. The restriction on alignment domains still applies.

Each domain gives the cleric access to a domain spell at each spell level he can cast, from 1st on up, as well as a granted power. The cleric gets the granted powers of both the domains selected.

With access to two domain spells at a given spell level, a cleric prepares one or the other each day in his domain spell slot. If a domain spell is not on the cleric spell list, a cleric can prepare it only in his domain spell slot.

\textbf{Spontaneous Casting}: A good cleric (or a neutral cleric of a good deity) can channel stored spell energy into healing spells that the cleric did not prepare ahead of time. The cleric can "lose" any prepared spell that is not a domain spell in order to cast any cure spell of the same spell level or lower (a cure spell is any spell with "cure" in its name).

An evil cleric (or a neutral cleric of an evil deity), can’t convert prepared spells to cure spells but can convert them to inflict spells (an inflict spell is one with "inflict" in its name).

A cleric who is neither good nor evil and whose deity is neither good nor evil can convert spells to either cure spells or inflict spells (player’s choice). Once the player makes this choice, it cannot be reversed. This choice also determines whether the cleric turns or commands undead.

\textbf{Chaotic, Evil, Good, and Lawful Spells}: A cleric can’t cast spells of an alignment opposed to his own or his deity’s (if he has one). Spells associated with particular alignments are indicated by the chaos, evil, good, and law descriptors in their spell descriptions.

\textbf{Turn or Rebuke Undead(Su)}: Any cleric, regardless of alignment, has the power to affect undead creatures by channeling the power of his faith through his holy (or unholy) symbol (see Turn or Rebuke Undead).

A good cleric (or a neutral cleric who worships a good deity) can turn or destroy undead creatures. An evil cleric (or a neutral cleric who worships an evil deity) instead rebukes or commands such creatures. A neutral cleric of a neutral deity must choose whether his turning ability functions as that of a good cleric or an evil cleric. Once this choice is made, it cannot be reversed. This decision also determines whether the cleric can cast spontaneous cure or inflict spells.

A cleric may attempt to turn undead a number of times per day equal to 3 + his Charisma modifier. A cleric with 5 or more ranks in Knowledge (religion) gets a +2 bonus on turning checks against undead.

\textbf{Bonus Feat}: At 4th level, a cleric gains a bonus feat. She can choose a metamagic feat, an item creation feat, or Spell Mastery. The cleric must still meet all prerequisites for a bonus feat, including caster level minimums.

This bonus feat is in addition to the feat that a character of any class gets from advancing levels. The cleric is not limited to the categories of item creation feats, metamagic feats, or Spell Mastery when choosing these feats.

\textbf{Bonus Languages}: A cleric’s bonus language options include Celestial, Abyssal, and Infernal (the languages of good, chaotic evil, and lawful evil outsiders, respectively). These choices are in addition to the bonus languages available to the character because of his race.