\textbf{\huge{Bard}}

\textbf{Hit Die}: d6

\textbf{Skills}: Appraise (Int), Balance (Dex), Bluff (Cha), Climb (Str), Concentration (Con), Craft (Int), Decipher Script (Int), Diplomacy (Cha), Disguise (Cha), Escape Artist (Dex), Gather Information (Cha), Heal (Wis), Hide (Dex), Jump (Str), Knowledge (all skills, taken individually) (Int), Listen (Wis), Move Silently (Dex), Perform (Cha), Profession (Wis), Sleight of Hand (Dex), Speak Language (None), Spellcraft (Int), Swim (Str), Tumble (Dex), Use Magic Device (Cha)

\textbf{Skills/Level}: 6 + Int modifier

\begin{center}
\begin{adjustwidth}{-4cm}{}
\begin{small}
\begin{tabular}{| c | c | c | c | c | c | c | c | c |}
\hline
LVL &BAB &F &R &W &Special &0 &1 &2 \\
\hline
1 &1 &0 &2 &2 &\makecell{Advanced Learning, Bardic Music, Bardic Knowledge, Skill\\ Focus (Perform), Inspire Courage +1, Weapon Finesse} &4/2 &- &- \\
2 &2 &0 &3 &3 &Fascinate &5/3 &2/1 &- \\
3 &3 &1 &3 &3 &Inspire Competence &6/3 &3/1 &- \\
4 &4 &1 &4 &4 &Countersong &6/3 &3/2 &2/1 \\
5 &5 &2 &4 &4 & &6/4 &4/3 &3/1 \\
6 &6/1 &2 &5 &5 &Suggestion &6/4 &4/3 &3/2 \\
\hline
\end{tabular}
\end{small}
\end{adjustwidth}
\end{center}

\textbf{Weapon and Armor Proficiency}: A bard is proficient with all simple and martial weapons. Bards are proficient with light armor, medium armor, and shields (except tower shields). A bard can cast bard spells while wearing light or medium armor without incurring the normal arcane spell failure chance. However wearing heavy armor or using a shield incurs a chance of arcane spell failure if the spell being cast has a somatic component. A multiclass bard still incurs the normal arcane spell failure chance for arcane spells received from other classes.

\textbf{Spells}: A bard casts arcane spells which are drawn from the bard spell list. He can cast any spell he knows without preparing it beforehand. To learn or cast a spell, a bard must have a Charisma score equal to at least 10 + the spell level. The Difficulty Class for a saving throw against a bard's spell is 10 + spell level + Charisma modifier.

Upon reaching 4th level, and at every Bard level after that, a Bard can choose to learn a new spell in place of one he already knows. 

\textbf{Advanced Learning}: At 1st, 2nd, 4th, and 6th level, a bard can gain an additional spell known, from a spell list outside of her own (should she choose to do so). The spell may be of a level no higher than that of the highest level spell the bard already knows. If a spell appears on multiple spell lists, use the lowest level version.

\textbf{Bardic Music}: Once per day per bard level, a bard can use his song or poetics to produce magical effects on those around him (usually including himself, if desired). While these abilities fall under the category of bardic music and the descriptions discuss singing or playing instruments, they can all be activated by reciting poetry, chanting, singing lyrical songs, singing melodies, whistling, playing an instrument, or playing an instrument in combination with some spoken performance. Each ability requires both a minimum bard level and a minimum number of ranks in the Perform skill to qualify; if a bard does not have the required number of ranks in at least one Perform skill, he does not gain the bardic music ability until he acquires the needed ranks.

Starting a bardic music effect is a standard action. Some bardic music abilities require concentration, which means the bard must take a standard action each round to maintain the ability. Even while using bardic music that doesn’t require concentration, a bard cannot cast spells, activate magic items by spell completion (such as scrolls), spell trigger (such as wands), or command word. Just as for casting a spell with a verbal component, a deaf bard has a 20\% chance to fail when attempting to use bardic music. If he fails, the attempt still counts against his daily limit.

\textbf{Bardic Knowledge}: A bard may make a special bardic knowledge check with a bonus equal to bard level + intelligence modifier + ranks in knowledge (history) divided by two. A successful bardic knowledge check will not reveal the powers of a magic item, but may give a hint as to its general function. A bard may not take 10 or 20 on this check.

\begin{center}
\begin{adjustwidth}{-4cm}{}
\begin{small}
\begin{tabular}{| l | l |}
\hline
DC &Type of Knowledge \\
\hline
10 &\makecell{Common, known by at least a substantial\\ minority of the local population.} \\
20 &Uncommon but available, known by only a few people legends. \\
25 &Obscure, known by few, hard to come by. \\
30 &\makecell{Extremely obscure, known by very few, possibly forgotten by most who once knew it,\\ possibly known only by those who don't understand the significance of the knowledge.} \\
\hline
\end{tabular}
\end{small}
\end{adjustwidth}
\end{center}

\textbf{Skill Focus (Perform)}: At 1st level a bard gains the feat skill focus for the skill perform. If the bard already has this feat, they can instead select a different feat for which they meet the prerequisites.

\textbf{Fascinate(Sp)}: A 2nd level bard with 3 or more ranks in a Perform skill can use his music or poetics to cause one or more creatures to become fascinated with him. Each creature to be fascinated must be within 90 feet, able to see and hear the bard, and able to pay attention to him. The bard must also be able to see the creature. The distraction of a nearby combat or other dangers prevents the ability from working. For every three levels a bard attains beyond 1st, he can target one additional creature with a single use of this ability.

\textbf{Inspire Courage(Su)}: A bard with 3 or more ranks in a Perform skill can use song or poetics to inspire courage in his allies (including himself), bolstering them against fear and improving their combat abilities. To be affected, an ally must be able to hear the bard sing. The effect lasts for as long as the ally hears the bard sing and for 5 rounds thereafter. An affected ally receives a +1 morale bonus on saving throws against charm and fear effects and a +1 morale bonus on attack and weapon damage rolls. Inspire courage is a mind-affecting ability.

\textbf{Weapon Finesse}: A bard may use her Dexterity modifier instead of Strength modifier on unarmed attack rolls. 

\textbf{Inspire Competence(Su)}: A bard of 3rd level or higher with 6 or more ranks in a Perform skill can use his music or poetics to help an ally succeed at a task. The ally must be within 30 feet and able to see and hear the bard. The bard must also be able to see the ally.

The ally gets a +2 competence bonus on skill checks with a particular skill as long as he or she continues to hear the bard’s music. Certain uses of this ability are infeasible. The effect lasts as long as the bard concentrates, up to a maximum of 2 minutes. A bard can’t inspire competence in himself. Inspire competence is a mind-affecting ability.

\textbf{Countersong(Su)}: A 4th level bard with 3 or more ranks in a Perform skill can use his music or poetics to counter magical effects that depend on sound (but not spells that simply have verbal components). Each round of the countersong, he makes a Perform check. Any creature within 30 feet of the bard (including the bard himself) that is affected by a sonic or language-dependent magical attack may use the bard’s Perform check result in place of its saving throw if, after the saving throw is rolled, the Perform check result proves to be higher. If a creature within range of the countersong is already under the effect of a noninstantaneous sonic or language-dependent magical attack, it gains another saving throw against the effect each round it hears the countersong, but it must use the bard’s Perform check result for the save. Countersong has no effect against effects that don’t allow saves. The bard may continue using the countersong for 10 rounds.

\textbf{Suggestion(Sp)}: A bard of 6th level or higher with 9 or more ranks in a Perform skill can make a suggestion (as the spell) to a creature that he has already fascinated. Using this ability does not break the bard’s concentration on the fascinate effect, nor does it allow a second saving throw against the fascinate effect.

Making a suggestion doesn’t count against a bard’s daily limit on bardic music performances. A Will saving throw (DC 10 + 1/2 bard’s level + bard’s Cha modifier) negates the effect. This ability affects only a single creature. Suggestion is an enchantment (compulsion), mind-affecting, language dependent ability.