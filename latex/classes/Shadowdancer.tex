\textbf{\huge{Shadowdancer}}

\textbf{Hit Die}: d6

\textbf{Skills}: Balance (Dex), Bluff (Cha), Decipher Script (Int), Diplomacy (Cha), Disguise (Cha), Escape Artist (Dex), Hide (Dex), Jump (Str), Listen (Wis), Move Silently (Dex), Perform (Cha), Profession (Wis), Search (Int), Sleight of Hand (Dex), Spot (Wis), Tumble (Dex), Use Rope (Dex)

\textbf{Skills/Level}: 4 + Int modifier

\begin{center}
\begin{adjustwidth}{-4cm}{}
\begin{small}
\begin{tabular}{| c | c | c | c | c | c |}
\hline
LVL &BAB &F &R &W &Special \\
\hline
1 &0 &0 &2 &0 &Hide in plain sight, shadow illusion \\
2 &1 &0 &3 &0 &Evasion, darkvision, shadow jump 20 ft., uncanny dodge \\
3 &2 &1 &3 &1 &Cloak of shadow \\
4 &3 &1 &3 &1 &Shadow jump 40 ft., shadow shield, summon shadow \\
5 &3 &1 &4 &1 &Improved uncanny dodge, shadow armor \\
6 &4 &2 &4 &2 &Shadow jump 80 ft., shadow skin \\
\hline
\end{tabular}
\end{small}
\end{adjustwidth}
\end{center}

\textbf{Weapon and Armor Proficiency}: Shadowdancers are proficient with the club, crossbow (hand, light, or heavy), dagger (any type), dart, mace, morningstar, quarterstaff, rapier, sap, shortbow (normal and composite), and short sword. Shadowdancers are proficient with light armor but not with shields.

\textbf{Hide in Plain Sight(Su)}: A 1st level shadowdancer can use the Hide skill even while being observed (as long as he is within 10 feet of some sort of shadow). He cannot however, hide in his own shadow. 

\textbf{Shadow Illusion(Sp)}: When a shadowdancer reaches 2nd level, she can create visual illusions. This ability’s effect is identical to that of the arcane spell silent image and may be employed at will. The save to disbelieve is 11 + the shadowdancer's Charisma modifier.

\textbf{Evasion(Ex)}:At 2nd level and higher, a shadowdancer can avoid even magical and unusual attacks with great agility. If she makes a successful Reflex saving throw against an attack that normally deals half damage on a successful save, she instead takes no damage. Evasion can be used only if the shadowdancer is wearing light armor or no armor. A helpless shadowdancer does not gain the benefit of evasion.

\textbf{Darkvision(Su)}: A 2nd level shadowdancer gains darkvision out to 60-feet, or 120-foot darkvision if he already had darkvision from another source.

\textbf{Shadow Jump(Su)}: At 2nd level, a shadowdancer gains the ability to travel between shadows as if by means of a dimension door spell. The limitations are that the magical transport must have line of sight, and must begin and end in an area with at least some shadow. The max distance a shadowdancer can jump begins at 20 ft., and doubles every two class levels, up to 80 ft. at level 6.

Making a jump requires a swift action, and can only be used once every 1d6 rounds.

\textbf{Uncanny Dodge(Ex)}: Starting at 2nd level, a shadowdancer retains her Dexterity bonus to AC (if any) regardless of being caught flat-footed or struck by an invisible attacker. (She still loses any Dexterity bonus to AC if immobilized.)

If a character gains uncanny dodge from a second class, the character automatically gains improved uncanny dodge (see below).

\textbf{Cloak of Shadow(Su)}: A shadowdancer in darkness gains a +2 untyped bonus on her saving throws, and concealment, even from enemies with darkvision.

\textbf{Shadow Shield(Su)}: A shadowdancer in darkness gains a +2 shield bonus to her Armor Class.

\textbf{Improved Uncanny Dodge(Ex)}: At 5th level, a shadowdancer can no longer be flanked. This defense denies rogues the ability to use flank attacks to sneak attack the shadowdancer. The exception to this defense is that a rogue at least four levels higher than the shadowdancer can flank her (and thus sneak attack her).

If a character gains uncanny dodge (see above) from a second class the character automatically gains improved uncanny dodge, and the levels from those classes stack to determine the minimum rogue level required to flank the character.

\textbf{Shadow Armor(Su)}: A shadowdancer in darkness gains a +2 dodge bonus to her Armor Class.

\textbf{Shadow Skin(Su)}: A shadowdancer in darkness gains damage reduction 3/-.

\textbf{\large{Summon Shadow}}

At 4th level, the shadowdancer gains the ability to conjure a humanoid companion out of shadow. This companion is controlled via a telepathic connection with the shadowdancer, in combat it does not have its own initiative score and instead acts on the turn of the shadowdancer.  

Summoning the companion requires a swift action, the companion may be placed in a square adjacent to the shadowdancer. Once summoned, the shadowdancer can dismiss the companion (causing it to disappear) as a free action. A dismissed companion can be summoned again in the same manner, even on the same turn it was dismissed.

If not dismissed, the companion remains until it is destroyed, its master dies, or its master sleeps. When a companion disappears, any ongoing spell effects (positive or negative) end. However damage sustained by the companion persists until its master takes a long rest, or it is healed by alternative means. If the companion is destroyed it cannot be summoned again until after its master takes a long rest.

The shadowdancer's companion is a medium, bipedal humanoid. It stands approximately 6 feet tall, and has no distinguishing features, save for a mouth. It possesses no senses of its own, instead relying on telepathic orders from the shadowdancer. It is not intelligent, it cannot speak, cast spells, activate magical items, or attempt any mental skill checks. The companion does not need to eat or drink, but it can consume potions or magical items and receive their benefits. For physical skill checks it is considered to have the same number of ranks invested as its shadowdancer master. Additionally the companion is considered proficient with the same varieties of weapons and armor as its master.

The companion can equip any gear it is given. It suffers the standard penalties for attempting to use weapons/armor with which it is not proficient. It cannot activate magic items, but gains any passive benefits from such items. If the companion dies, is dismissed, or otherwise disappears, all equipped gear or carried items fall in a pile on the space the companion occupied. 

The companion shares the traits of undead creatures, except that unlike undead the companion can be healed through conventional means. As the shadowdancer advances in level the shadowdancer's companion increases in strength, specific stats are given on the table below.

\begin{center}
\begin{adjustwidth}{-4cm}{}
\begin{small}
\begin{tabular}{| l | l | l | l |}
\hline
Shadowdancer Companion &Level 4 &Level 5 &Level 6 \\
\hline
Size/Type &Medium Humanoid & & \\
Hit Dice: &3d12 (19) &4d12 (26) &5d12 (32) \\
Speed: &20 ft. &30 ft. &40 ft. \\
Armor Class: &14 (+2 dex, +2 natural) &15 (+2 dex, +3 natural) &16 (+3 dex, +3 natural) \\
Base Attack/Grapple: &+4/+6 &+5/+8 &+6/+9 \\
Attack: &Shadow Weapon +6 melee (1d6+2) &Shadow Weapon +8 melee (1d6+3) &Shadow Weapon +9 melee (1d6+3) \\
Full Attack: &Shadow Weapon +6 melee (1d6+2) &Shadow Weapon +8 melee (1d6+3) &Shadow Weapon +9/+4 melee (1d6+3) \\
Space/Reach: &5 ft./5 ft. & & \\
Special Attacks: & & & \\
Special Qualities: &Undead traits (except for healing) & & \\
Saves: &Fort +1, Ref +5, Will +1 &Fort +1, Ref +6, Will +1 &Fort +2, Ref +6, Will +2 \\
Abilities: &Str 14, Dex 14 &Str 16, Dex 14 &Str 16, Dex 16 \\
Feats: &Bonus Feat &Evasion &Bonus Feat \\
\hline
\end{tabular}
\end{small}
\end{adjustwidth}
\end{center}

\textbf{Attack}: The shadowdancer companion's default means of attack is by conjuring a weapon out of shadow. This weapon can be conjured or dismissed as a free action. It deals 1d6 damage with a critical range of 19-20/x2. The type of damage dealt (bludgeoning, piercing, or slashing) is selected by the shadowdancer, and the weapon is treated as magical for the purpose of overcoming damage reduction. If the companion is disarmed or voluntarily lets go of this weapon it disappears. The companion can only have one such weapon conjured at a time.

\textbf{Bonus Feat}: At shadowdancer levels 4 and 6 the companion gains a bonus feat for which it meets the prerequisites. These feats cannot allow the companion to perform any action it could not normally perform (for example cast a spell, or speak).