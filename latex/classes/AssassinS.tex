\textbf{\huge{Assassin}}

\textbf{Hit Die}: d6

\textbf{Skills}: Balance (Dex), Bluff (Cha), Climb (Str), Concentration (Con), Craft (Int), Disable Device (Int), Disguise (Cha), Escape Artist (Dex), Gather Information (Cha), Heal (Wis), Hide (Dex), Intimidate (Cha), Jump (Str), Knowledge (all skills, taken individually) (Int), Listen (Wis), Move Silently (Dex), Perform (Cha), Profession (Wis), Search (Int), Sense Motive (Wis), Sleight of Hand (Dex), Spellcraft (Int), Spot (Wis), Swim (Str), Tumble (Dex), Use Magic Device (Cha)

\textbf{Skills/Level}: 6 + Int modifier

\begin{center}
\begin{small}
\begin{tabular}{| c | c | c | c | c | c | c | c | c |}
\hline
LVL &BAB &F &R &W &Special &0 &1 &2 \\
\hline
1 &0 &0 &2 &0 &Death Attack, Personal Immunity, Poison Use &4/2 &- &- \\
2 &1 &0 &3 &0 &Uncanny Dodge &5/3 &2/0 &- \\
3 &2 &1 &3 &1 &Hide in Plain Sight &6/3 &3/1 &2/0 \\
4 &3 &1 &4 &1 &Cloak of Discretion, Skill Mastery &6/3 &3/2 &3/1 \\
5 &3 &1 &4 &1 &Trapfinding, Trapmaking, Nerve of the Killer &6/3 &4/3 &3/2 \\
6 &4 &2 &5 &2 &Palm Weapon, Poisonmaster &6/3 &4/3 &3/2 \\
\hline
\end{tabular}
\end{small}
\end{center}

\textbf{Spellcasting}: An assassin is an Arcane Spellcaster with the same spells per day and spells known progression as a Bard, except that he gains no more than three spell slots per level. An assassin’s spells known may be chosen from the Sorcerer/Wizard list, and must be from the schools of Divination, Illusion, or Necromancy. To cast an assassin spell, she must have an Intelligence at least equal to 10 + the spell level. The DC of the assassins’ spells is Intelligence based and the bonus spells are Intelligence based.

\textbf{Weapon and Armor Proficiency}: assassins are proficient with all simple and martial weapons. assassins are proficient with all light heavy armor.

\textbf{Death Attack}: An assassin may spend a full-round action to study an opponent. If she does so, her next attack is a Death Attack if she makes it within 1 round. A Death Attack inflicts a number of extra dice of damage equal to her assassin level plus two dice, but only if the target is denied its Dexterity Bonus to AC against that attack. Special attacks such as a coup de grace may be a Death Attack. If a character has both sneak attack and death attack, they stack if the character meets the requirements of both. An assassin may load a crossbow simultaneously with his action to study his target if he has a Base Attack Bonus of +1 or more.

\textbf{Personal Immunity}: At each level an assassin may choose a poison to be immune to. At 6th level the assassin gains immunity to all poison.

\textbf{Poison Use}: An assassin may prepare, apply, and use poison without any chance of poisoning herself.

\textbf{Uncanny Dodge}: Starting at 2nd level, an assassin can react to danger before his senses would normally allow him to do so. He retains his Dexterity bonus to AC (if any) even if he is caught flat-footed or struck by an invisible attacker. However, he still loses his Dexterity bonus to AC if immobilized. 

If an assassin already has uncanny dodge from a different class he automatically gains improved uncanny dodge instead.

\textbf{Hide in Plain Sight}: A 3rd level assassin can hide in unusual locations and may hide in areas without cover or concealment without penalty. An assassin may even hide while being observed. This ability does not remove the -10 penalty for moving at full speed or the -20 penalty for running or fighting.

\textbf{Cloak of Discretion(Su)}: A 4th level assassin is protected by a constant nondetection effect, with a caster level equal to his character level.

\textbf{Skill Mastery}: A 5th level assassin may always take 10 on the following skills, climb, disable device, hide, move silently, search, spellcraft, use magic device, use rope, or swim.

\textbf{Trapfinding}: At 5th level, assassins can use the Search skill to locate traps when the task has a Difficulty Class higher than 20. Finding a nonmagical trap has a DC of at least 20, or higher if it is well hidden. Finding a magic trap has a DC of 25 + the level of the spell used to create it. Assasins can use the Disable Device skill to disarm magic traps. A magic trap generally has a DC of 25 + the level of the spell used to create it. An assassin who beats a trap’s DC by 10 or more with a Disable Device check can study a trap, figure out how it works, and bypass it (with her party) without disarming it.

\textbf{Trapmaking}: A 5h level assassin learns to build simple mecahnical traps out of common materials. As long as he has access to ropes, flexible material like green wood, and weapon-grade materials like sharpened wooden sticks or steel weapons, he can build an improved trap in 10 minutes. He can build any non-magical trap on the "CR 1" trap list that doesn't involve a pit. These traps have a search DC equal to 20 + the assassin's level, have a BAB equal to his own, and are always single-use traps. He may add poison to these traps, if he has access to it, but it will dry out in an hour.

\textbf{Nerves of the Killer}:  A 5th level assassin gains a limited immunity to compulsion and charm effects. While studying a target for a Death Attack, and for one round afterward, he counts as if he were within a protection from evil effect. This does not confer a deflection bonus to AC.

\textbf{Palm Weapon(Su)}: A 6th level assassin may conceal weapons with supernatural skill. Any weapon concealed with Sleight of Hand cannot be found with Divination magic.

\textbf{Poisonmaster}: A 6th level assassin can create short term poisons. By expending an entire healer's kit worth of materials and an hour of time, he can synthesize short term poisons which degrade to uselessness after one week.

These poisons deal 3d6 damage as a primary effect, and should the target fail its fortitude save, 2d6 ability score damage as a secondary effect. The save DC is 16 plus the assassin's Intelligence Modifier. The type of ability score damage is selected when the poison is created. If a target is affected by multiple poisons targeting the same ability score the ability score damage does not stack.
