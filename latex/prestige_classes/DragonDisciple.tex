\textbf{\huge{Dragon Disciple}}

\textbf{Hit Die}: d12

\textbf{Skills}: Concentration (Con), Craft (Int), Diplomacy (Cha), Escape Artist (Dex), Gather Information (Cha), Knowledge (all skills, taken individually) (Int), Listen (Wis), Profession (Wis), Search (Int), Speak Language (None), Spellcraft (Int), Spot (Wis)

\textbf{Skills/Level}: 2 + Int modifier

\textbf{\large{Requirements}}

To qualify to become a dragon disciple, a character must fulfill all the following criteria.

\textbf{Race}: Any nondragon (cannot already be half-dragon).

\textbf{Spellcasting}: Ability to cast arcane spells without preparation

\textbf{Special}: The player chooses a dragon variety when taking the first level of this prestige class. Taken dragon disciple levels cannot be retrained. Dragon disciple levels stack with sorcerer levels for the purpose of determining caster level.

\begin{center}
\begin{adjustwidth}{-4cm}{}
\begin{small}
\begin{tabular}{| c | c | c | c | c | c | c |}
\hline
LVL &BAB &F &R &W &Special &Spellcasting \\
\hline
1 &1 &2 &0 &2 &Natural Armor Increase (+1) &- \\
2 &2 &3 &0 &3 &Ability Boost (Str +2), breath weapon (2d8), claws and bite &- \\
3 &3 &3 &1 &3 &\makecell{Ability Boost (Dex +2), Blindsense\\ 30 ft, natural armor increase (+2)} &+1 level of existing spellcasting class \\
4 &4 &4 &1 &4 &Ability Boost (Con +2), breath weapon (3d8) &- \\
5 &5 &4 &2 &4 &Draconic Aspect, Natural armor increase (+3) &+1 level of existing spellcasting class \\
\hline
\end{tabular}
\end{small}
\end{adjustwidth}
\end{center}

\textbf{Weapon and Armor Proficiency}: Dragon disciples gain no proficiency with any weapon or armor.

\textbf{Spells per Day}: When a dragon disciple reaches levels 3 and 5, the character gains new spells per day as if he had also gained a level in a spellcasting class he belonged to before adding the prestige class. He does not, however, gain any other benefit a character of that class would have gained, except for an increased effective level of spellcasting. If a character had more than one spellcasting class before becoming an dragon disciple, he must decide to which class he adds the new level for purposes of determining spells per day.

\textbf{Natural Armor Increase}: At 1st, 3rd, and 5th level, a dragon disciple gains an increase to the character’s existing natural armor (if any). As his skin thickens, a dragon disciple takes on more and more of his progenitor’s physical aspect.

\textbf{Ability Boost(Ex)}: As a dragon disciple gains levels in this prestige class, his ability scores increase.

These increases stack and are gained as if through level advancement.

\textbf{Breath Weapon(Su)}: At 2nd level, a dragon disciple gains a minor breath weapon. The type and shape depend on the dragon variety whose heritage he enjoys (see below). Regardless of the ancestor, the breath weapon deals 2d8 points of damage of the appropriate energy type. At 4th level the damage increases to 3d8.

Regardless of its strength, the breath weapon can be used only once every 1d4 rounds. Use all the rules for dragon breath weapons except as specified here.

The DC of the breath weapon is 10 + class level + Con modifier.

A line-shaped breath weapon is 5 feet high, 5 feet wide, and 60 feet long. A cone-shaped breath weapon is 30 feet long.

\begin{center}
\begin{adjustwidth}{-4cm}{}
\begin{small}
\begin{tabular}{| l | l |}
\hline
Dragon Variety &Breath Weapon \\
\hline
Black &Line of acid \\
Blue &Line of lightning \\
Green &Cone of corrosive gas (acid) \\
Red &Cone of fire \\
White &Cone of cold \\
Brass &Line of fire \\
Bronze &Line of lightning \\
Copper &Line of acid \\
Gold &Cone of fire \\
Silver &Cone of cold \\
\hline
\end{tabular}
\end{small}
\end{adjustwidth}
\end{center}

\textbf{Claws and Bite(Ex)}: At 2nd level, a dragon disciple gains claw and bite attacks if he does not already have them. Use the values below or the disciple’s base claw and bite damage values, whichever are greater.

A dragon disciple is considered proficient with these attacks. When making a full attack, a dragon disciple uses his full base attack bonus with his bite attack but takes a -5 penalty on claw attacks. The Multiattack feat reduces this penalty to only -2.

\begin{center}
\begin{adjustwidth}{-4cm}{}
\begin{small}
\begin{tabular}{| l | l | l |}
\hline
Size &Bite Damage &Claw Damage \\
\hline
Small &1d4 &1d3 \\
Medium &1d6 &1d4 \\
Large &1d8 &1d6 \\
\hline
\end{tabular}
\end{small}
\end{adjustwidth}
\end{center}

\textbf{Blindsense(Ex)}: At 3rd level, the dragon disciple gains blindsense with a range of 30 feet. Using nonvisual senses the dragon disciple notices things it cannot see. He usually does not need to make Spot or Listen checks to notice and pinpoint the location of creatures within range of his blindsense ability, provided that he has line of effect to that creature.

Any opponent the dragon disciple cannot see still has total concealment against him, and the dragon disciple still has the normal miss chance when attacking foes that have concealment. Visibility still affects the movement of a creature with blindsense. A creature with blindsense is still denied its Dexterity bonus to Armor Class against attacks from creatures it cannot see.

\textbf{Draconic Aspect}: A 5th level dragon disciple's appearance is forever altered, their skin gains a smooth, metallic quality, and a light tinge matching the color of their chosen dragon variety. Their eyes change to a deep golden color. 

They gain a +2 to charisma based skill checks, +1 to saves against paralysis, sleep, or poison, and darkvision out to 60 feet (120 feet if they already had darkvision from another source).